%%%%%%%%%%%%%%%%%%%%%%%%%%%%%%%%%%%%%%%%%%%%%%%%%%%%%%%%%%%%%%%%%%%%%%%%%%%%%%%

% loaduju některé balíčky -----------------------------------------------------
\usepackage{amsmath}
\usepackage{mathtools}
\usepackage{physics}
\usepackage{graphicx}
\usepackage{tocloft}
\usepackage{enumerate}
\usepackage{eso-pic}
\usepackage[czech]{babel}
\usepackage{csquotes}
\usepackage{amsfonts}
\usepackage{amssymb}
\usepackage{amsthm}
\usepackage{float}
\usepackage{bm}
\usepackage[bottom]{footmisc}
\usepackage{caption}
\usepackage{array}
\usepackage{multirow}
\usepackage{color}
\usepackage{adjustbox}
\usepackage{makecell}


% české uvozovky --------------------------------------------------------------
\DeclareQuoteAlias{german}{czech}
\MakeOuterQuote{"}

% přejmenovávám popisek obsahu na "Obsah" -------------------------------------
\renewcommand{\contentsname}{Obsah}

% přejmenovávám popisky obrázků na "Obr." -------------------------------------
\renewcommand{\figurename}{Obr.}

% přejmenovávám popisky tabulek na "Tab." -------------------------------------
\renewcommand\tablename{Tab.}

% upravuji formát obsahu ------------------------------------------------------
\renewcommand{\cftsecleader}{\cftdotfill{\cftdotsep}}

% definuji prostředí nečíslovaného lemmatu ------------------------------------
\newtheorem*{lemma*}{Lemma}

% definuji prostředí nečíslované definice -------------------------------------
\newtheorem*{mydef*}{Definice}
\newtheoremstyle{named}{}{}{\itshape}{}{\bfseries}{.}{.5em}{\thmnote{#3's }#1}
\theoremstyle{named}
\newtheorem*{namedmydef}{Definice}

% definuji dolní a horní celou část -------------------------------------------
\DeclarePairedDelimiter\ceil{\lceil}{\rceil}
\DeclarePairedDelimiter\floor{\lfloor}{\rfloor}

% definuji arg max ------------------------------------------------------------
\DeclareMathOperator*{\argmax}{arg\,max}


%%%%%%%%%%%%%%%%%%%%%%%%%%%%%%%%%%%%%%%%%%%%%%%%%%%%%%%%%%%%%%%%%%%%%%%%%%%%%%%
